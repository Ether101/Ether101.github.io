\documentclass{article}
\usepackage{amsmath, amssymb, amsthm}
\usepackage{graphicx}
\usepackage{hyperref}

\title{Lab Report for Resume Portfolio}
\author{}
\date{}

\begin{document}

\maketitle

\section*{Links}
\begin{itemize}
    \item \href{https://ether101.github.io/}{Portfolio Webpage: My Portfolio}
    \item \href{https://github.com/Ether101}{GitHub Repository: GitHub Repository}
\end{itemize}

\section*{Portfolio Website and GitHub Repository Format}

\subsection*{Website Structure}
The portfolio website is divided into several sections, each representing a different aspect:
\begin{itemize}
    \item \textbf{Header (Navigation)}: Provides links to the About, Experience, Projects, and Contact sections.
    \item \textbf{Introduction}: A short introduction with a profile picture.
    \item \textbf{About}: Section to showcase personal background.
    \item \textbf{Experience}: Summaries of professional experience and skills.
    \item \textbf{Projects}: Showcases various personal projects and links them to the GitHub Repository.
    \item \textbf{Contact}: Contains contact information and a form for visitors to reach out.
    \item \textbf{Footer}: Displays copyright information.
\end{itemize}

\subsection*{GitHub Repository Structure}
The GitHub repository contains the following files and directories:
\begin{itemize}
    \item \textbf{index.html}: Main HTML file containing the structure of the webpage.
    \item \textbf{style.css}: Contains all the styles used in the webpage.
    \item \textbf{mediaqueries.css}: CSS rules for the responsive design.
    \item \textbf{script.js}: JavaScript functions for interactive elements.
    \item \textbf{assets/}: Directory containing images and the resume used in the webpage.
    \item \textbf{README.md}: Provides the LaTeX code of this PDF file.
\end{itemize}

\subsection*{Design Decisions}
\textbf{Clean and Simple Layout}: The design focuses on content within the webpage without overwhelming visitors with overly glamorous effects. Using ample white space and a consistent color scheme helps maintain a professional look.

\textbf{Responsive Design}: Ensures that the portfolio looks good on all devices. The \texttt{mediaqueries.css} adjusts the layout based on screen size.

\textbf{Typography (fonts)}: The Poppins font from Google Fonts is used throughout the webpage to provide a modern and readable look.

\subsection*{Interactive Elements}
\textbf{Navigation Menu}: Styled to be both functional and visually appealing, improving the interface and providing a technical touch. Flexbox is used to align the items and space within the navigation bar. Hover effects are added to links to enhance interactivity.

\textbf{Buttons}: Smooth hover effects are achieved with transitions. Padding and border radius are added to make buttons more appealing and clickable. The code for these features is found in the \texttt{style.css} file.

\textbf{Profile}: Includes a centered layout with a profile picture beside the introduction and one under the 'About Me' section. Flexbox is used to align images and provide spacing between elements for better readability.

\section*{Tools and Technology}
A combination of computing languages was used, with the majority of the site done in HTML and CSS:
\begin{itemize}
    \item \textbf{HTML}: Used for the content of the portfolio (\texttt{index.html}).
    \item \textbf{CSS}: Used for styling and ensuring website responsiveness (\texttt{style.css} and \texttt{mediaqueries.css}).
    \item \textbf{JavaScript}: Used for adding interactive features such as toggling the navigation menu.
    \item \textbf{GitHub}: Used for version control, hosting the webpage through Pages, and organizing files in a repository.
    \item \textbf{Virtual Studio Code}: Primary text editor used for building the webpage. Extensions like Live Server allowed viewing the webpage during development.
    \item \textbf{Git}: Used for pulling, pushing, and committing changes to the GitHub repository.
\end{itemize}

\section*{Reflections}
\subsection*{Strengths}
\begin{itemize}
    \item \textbf{Responsive Design}: The website looks good on all devices.
    \item \textbf{Clean Layout}: The design is simple and professional.
    \item \textbf{Easy Navigation}: The navigation menu is user-friendly and responsive.
\end{itemize}

\subsection*{Weaknesses}
\begin{itemize}
    \item \textbf{Limited Interactivity}: The website does not have many interactive elements.
    \item \textbf{Basic Styling}: Although the website appears clean, it could benefit from more styling enhancements.
\end{itemize}

\subsection*{Future Improvements}
\begin{itemize}
    \item \textbf{Expand Content}: Add more projects and provide detailed descriptions.
    \item \textbf{Add Animations}: Use subtle animations to improve user experience.
    \item \textbf{Improve Accessibility}: Add features to make the website more accessible for users with disabilities.
\end{itemize}

\begin{figure}
    \centering
    \includegraphics[width = 1.2\textwidth
    ]{Images/Interactive_Elements_Image.png}
    \caption{Interactive Elements code snippet}
    \label{fig:enter-label}
\end{figure}
\begin{figure}
    \centering
    \includegraphics[width = 1.2\textwidth
    ]{Images/Navigation_Menu_Image.png}
    \caption{Navigation Menu code snippet}
    \label{fig:enter-label}
\end{figure}
\begin{figure}
    \centering
    \includegraphics[width = 1.2\textwidth
    ]{Images/responsive_design.png}
    \caption{Responsive design snippet}
    \label{fig:enter-label}
\end{figure}
\begin{figure}
    \centering
    \includegraphics[width = 1.2\textwidth
    ]{Images/Typography_Image.png}
    \caption{Typography Code snippet}
    \label{fig:enter-label}
\end{figure}
\end{document}
```